\documentclass{article}
\usepackage[a4paper]{geometry}
\usepackage[utf8]{inputenc} 
\usepackage{indentfirst}
%list code
\usepackage{listings}
%content include references
\usepackage[nottoc,numbib]{tocbibind}
% table of contents clickabel
\usepackage{hyperref}
% make file system neat and easier to manage
\usepackage{subfiles}
% packages
\usepackage{csquotes}
\usepackage{arydshln}
\usepackage{dsfont}
\usepackage{polynom}
\usepackage{empheq}
\usepackage{calc}  
\usepackage{enumitem}
\usepackage{graphicx}
\usepackage{systeme}
\usepackage{mathtools}
\usepackage{tikz}
\usepackage{amsfonts, mathrsfs, bm, amsmath, amssymb, bbm, amsthm}
\usepackage{ifthen}
\usepackage{enumerate}
% Make the preview part more eye friendly
\usepackage{xcolor}
\pagecolor[rgb]{1,1,1} % background
\color[rgb]{0,0,0} % foreground
% new command
\DeclareMathOperator*{\argmax}{arg~max}
\DeclareMathOperator*{\argmin}{arg~min}
% theorem block
\newtheorem{theorem}{Theorem}[section]
\newtheorem{note}{Note}
% cross page equation
\allowdisplaybreaks

\begin{document}
    \section{EM Algorithm}
        The goal of the EM algorithm is to find maximum likelihood solutions for models having latent variables. Consider a probabilistic model in which we collectively denote all of the observed variabels by \textbf{X} and all of the hidden variables by \textbf{Z}. The joint distribution $p(\mathbf{X},\mathbf{Z}|\boldsymbol{\theta})$ is governed by a set of parameters denoted $\boldsymbol{\theta}$. Our goal is to maximize the likelihood function that is given by
        \[
            p(\mathbf{X}|\boldsymbol{\theta})=\sum_{\mathbf{Z}}p(\mathbf{X}, \mathbf{Z}|\boldsymbol{\theta})
        \]
        This method based on the assumption that the computation of $p(\mathbf{X}, \mathbf{Z}|\boldsymbol{\theta})$ is easier than $p(\mathbf{X}|\boldsymbol{\theta})$.
        Let $q(\mathbf{Z})$ be the distribution of $\mathbf{Z}$. Then is obvious that
        \[
            \ln p(\mathbf{X}|\boldsymbol{\theta})=\sum_{\mathbf{Z}}q(\mathbf{Z})\ln p(\mathbf{X}|\boldsymbol{\theta})
        \]
        We've known that
        \[
            \ln p(\mathbf{X},\mathbf{Z}|\boldsymbol{\theta}) = \ln p(\mathbf{Z}|\mathbf{X},\boldsymbol{\theta})+ \ln p(\mathbf{X}|\boldsymbol{\theta})
        \]
        Then
        \begin{align*}
            \ln p(\mathbf{X}|\boldsymbol{\theta})&=\sum_{\mathbf{Z}}\ln q(\mathbf{Z})p(\mathbf{X}|\boldsymbol{\theta})\\
            &=\sum_{\mathbf{Z}}q(\mathbf{Z})\ln\left\{\frac{p(\mathbf{X},\mathbf{Z}|\boldsymbol{\theta})}{q(\mathbf{Z})}\right\}-\sum_{\mathbf{Z}}q(\mathbf{Z})\ln\left\{\frac{p(\mathbf{Z}|\mathbf{X},\boldsymbol{\theta})}{q(\mathbf{Z})}\right\}\\
            &\equiv\mathcal{L}(q,\boldsymbol{\theta})+\mathrm{KL}(q\|p)
        \end{align*}
        Since $\mathrm{KL}(q\|p)\geq0$, $\mathcal{L}(q,\boldsymbol{\theta})$ is the lower bound of $\ln p(\mathbf{X}|\boldsymbol{\theta})$, i.e.
        \[
            \mathcal{L}(q,\boldsymbol{\theta})\leq\ln p(\mathbf{X}|\boldsymbol{\theta})
        \]
        Note that $\mathrm{KL}(q\|p)=0$ if and only if $q(\mathbf{Z})=p(\mathbf{Z}|\mathbf{X},\boldsymbol{\theta})$.
        
        The EM algorithm is a two-stage iterative optimization technique for finding maximum likelihood solutions.
        \subsection{E Step}
        Let $\boldsymbol{\theta}^{\mathrm{old}}$ be the current parameter vector. Maximize $\mathcal{L}(q,\boldsymbol{\theta})$ with respect to $q(\mathbf{Z})$ with holding $\boldsymbol{\theta}^{\mathrm{old}}$ fixed. $\mathcal{L}(q,\boldsymbol{\theta})$ will reach its maximum $\ln p(\mathbf{X}|\boldsymbol{\theta})$ when $q(\mathbf{Z})=p(\mathbf{Z}|\mathbf{X},\boldsymbol{\theta}^\mathrm{old})$. That is,
        \begin{align*}
            \max\mathcal{L}(q,\boldsymbol{\theta})&=\sum_{\mathbf{Z}}p(\mathbf{Z}|\mathbf{X},\boldsymbol{\theta}^{\mathrm{old}})\ln p(\mathbf{X},\mathbf{Z}|\boldsymbol{\theta}^\mathrm{old}) - \sum_{\mathbf{Z}}p(\mathbf{Z}|\mathbf{X},\boldsymbol{\theta}^{\mathrm{old}})\ln p(\mathbf{Z}|\mathbf{X},\boldsymbol{\theta}^{\mathrm{old}})
        \end{align*}
        \subsection{M Step}
        Maximize $\mathcal{L}(q,\boldsymbol{\theta})$ with respect to $\boldsymbol{\theta}$ with holding $q(\mathbf{Z})$ fixed. 
        Here we get new parameter $\boldsymbol{\theta}^{\mathrm{new}}$. This will cause $\mathcal{L}(q,\boldsymbol{\theta})$ increase and then cause $\ln p(\mathbf{X}|\boldsymbol{\theta})$ increase subsequently. Furthermore, the $q(\mathbf{Z})$ in this step is determined in the previous step, $\mathrm{KL}(q,\boldsymbol{\theta}^{\mathrm{new}})$ will be nonzero. Hence $\ln p(\mathbf{X}|\boldsymbol{\theta})$ increases more than its lowe bound $\mathcal{L}(q,\boldsymbol{\theta})$. From above we have,
        \begin{align*}
            \mathcal{L}(q,\boldsymbol{\theta})&=\sum_{\mathbf{Z}}p(\mathbf{Z}|\mathbf{X},\boldsymbol{\theta}^\mathrm{old})\ln p(\mathbf{X},\mathbf{Z}|\boldsymbol{\theta})-\sum_{\mathbf{Z}}p(\mathbf{Z}|\mathbf{X},\boldsymbol{\theta}^\mathrm{old})\ln p(\mathbf{Z}|\mathbf{X},\boldsymbol{\theta}^\mathrm{old})\\
            &=\mathcal{Q}(\boldsymbol{\theta},\boldsymbol{\theta}^\mathrm{old}) + \mathrm{constant}
        \end{align*}
        Since $p(\mathbf{Z}|\mathbf{X},\boldsymbol{\theta}^\mathrm{old}) = q(\mathbf{Z})$, the last term is a constant. This tells us that we actually maximize the expectation of the complete-data log likelihood.
\end{document}
