\documentclass{article}
\usepackage[a4paper]{geometry}
\usepackage[utf8]{inputenc} 
\usepackage{indentfirst}
%list code
\usepackage{listings}
%content include references
\usepackage[nottoc,numbib]{tocbibind}
% table of contents clickabel
\usepackage{hyperref}
% make file system neat and easier to manage
\usepackage{subfiles}
% packages
\usepackage{csquotes}
\usepackage{arydshln}
\usepackage{dsfont}
\usepackage{polynom}
\usepackage{empheq}
\usepackage{calc}  
\usepackage{enumitem}
\usepackage{graphicx}
\usepackage{systeme}
\usepackage{mathtools}
\usepackage{tikz}
\usepackage{amsfonts, mathrsfs, bm, amsmath, amssymb, bbm, amsthm}
\usepackage{ifthen}
\usepackage{enumerate}
% Make the preview part more eye friendly
\usepackage{xcolor}
\pagecolor[rgb]{1,1,1} % background
\color[rgb]{0,0,0} % foreground
% new command
\DeclareMathOperator*{\argmax}{arg~max}
\DeclareMathOperator*{\argmin}{arg~min}
% theorem block
\newtheorem{theorem}{Theorem}[section]
\newtheorem{note}{Note}
% cross page equation
\allowdisplaybreaks

\begin{document}
    \section{Principal Component Analysis(PCA)}
        \subsection{Settings}
            Let $\mathbf{X} = \{\mathbf{x}_i\}_{i=1}^n\in\mathbb{R}^p$ be a sample of $n$ observations of $p$ dimensional vectors. The first principal component of this sample is a real variable transformed from $X$
            \[
                \mathbf{z}_1 = \mathbf{X}\mathbf{a}_1^T
            \]
            where the vector $\mathbf{a}_1=(a_{11}, a_{21},\dots,a_{p1})\in\mathbb{R}^p$ is chosen so that $\|\mathbf{a}_1\|_2=1$ and $var[\mathbf{z}_1]$ is maximized.\\
            The $k^{\mathrm{th}}$ principal component of the sample is
            \[
                \mathbf{z}_k = \mathbf{X}\mathbf{a}_k^T,\quad k=2,\dots,p
            \]
            where the vector $\mathbf{a}_k=(a_{1k},a_{2k},\dots,a_{pk})\in\mathbb{R}^p$ is chosen so that $\|\mathbf{a}_k\|_2=1$, $var[\mathbf{z}_k]$ is maximized and $cov[\mathbf{z}_k,\mathbf{z}_l]=0$ for $k>l\geq1$.
            
            Geometrically, at first step, PCA finds a unit vector $\mathbf{a}_1\in\mathbb{R}^p$ that the projections of sample points on it has maximum variance. $\mathbf{z}_1$ is actually a variable of projection length of sample points on $\mathbf{a}_1$. Secondly, PCA finds the second component $\textbf{z}_2$ in the same logic but with further restriction that $\mathbf{z}_2$ has no relation with $\mathbf{z}_1$, i.e. $cov[\mathbf{z}_2,\mathbf{z}_1]=0$.
        
        \subsection{Computations}
            To find $\mathbf{a}_1$, first note that
            \begin{align*}
                var[\textbf{z}_1]=&\mathrm{E}[\mathbf{z}_1^2]-\mathrm{E}[\mathbf{z}_1]^2\\
                                =&\sum_{i,j=1}^pa_{i1}a_{j_1}\mathrm{E}[X_iX_j]-\sum_{i,j=1}^pa_{i1}a_{j1}\mathrm{E}[X_i]\mathrm{E}[X_j]\\
                                =&\sum_{i,j=1}^pa_{i1}a_{j1}cov[X_i,X_j]\\
                                =&(a_{11},a_{21},\dots,a_{p1})
                                \begin{pmatrix}
                                    cov[X_1,X_1] & cov[X_1,X_2] & \dots & cov[X_1,X_p]\\
                                    cov[X_2,X_1] & cov[X_2,X_2] & \dots & cov[X_2,X_p]\\
                                    \vdots & \vdots & \ddots & \vdots \\
                                    cov[X_p,X_1] & cov[X_p,X_2] & \dots & cov[X_p,X_p]
                                \end{pmatrix}
                                \begin{pmatrix}
                                    a_{11} \\ a_{21} \\ \vdots \\ a_{p1}
                                \end{pmatrix}\\
                                =&\mathbf{a}_1\Sigma\mathbf{a}_1^T
            \end{align*}
            where $\Sigma$ is the covariance matrix of $X$. The problem becomes
            \[
                \max(\mathbf{a}_1\Sigma\mathbf{a}_1^T) \quad s.t.\quad \|\mathbf{a}_1\|_2=1
            \]
            Let $\lambda$ be Lagrange multiplier and let $g(\mathbf{a}_1)=\mathbf{a}_1\Sigma\mathbf{a}_1^T-\lambda(\mathbf{a}_1\mathbf{a}_1^T-1)$. Then we need
            \[
                \nabla g(\mathbf{a}_1)=\mathbf{a}_1\Sigma^T-\lambda\mathbf{a}_1=(\Sigma\mathbf{a}_1^T-\lambda\mathbf{a}_1^T)^T=0
            \]
            Therefore, $\mathbf{a}_1^T$ is the eigenvector of $\Sigma$ corresponding to eigenvalue $\lambda\equiv\lambda_1$. Furthermore, we have maximized $var[\mathbf{z}_1]=\mathbf{a}_1\Sigma\mathbf{a}_1^T=\lambda_1$, this means $\lambda_1$ is the largest eigenvalue of $\Sigma$.
            
            For the second component, note that $cov[\mathbf{z}_2, \mathbf{z}_1]=\mathbf{a}_2\Sigma\mathbf{a}_1^T=\mathbf{a}_2\lambda_1\mathbf{a}_1^T=0$. For another multipliers $\phi$ and $\lambda$ and another $g(\mathbf{a}_2)=\mathbf{a}_2\Sigma\mathbf{a}_2^T-\lambda(\mathbf{a}_2\mathbf{a}_2^T-1)-\phi\lambda_1\mathbf{a}_2\mathbf{a}_1^T$, we need
            \[
                \nabla g(\mathbf{a}_2) = (\Sigma\mathbf{a}_2^T-\lambda\mathbf{a}_2^T-\phi\lambda_1\mathbf{a}_1^T)^T=0
            \]
            Multiply $\mathbf{a}_1$ both side, we get
            \begin{align*}
                &\mathbf{a}_1\Sigma\mathbf{a}_2^T - \mathbf{a}_1\lambda\mathbf{a}_2^T-\phi\lambda_1\mathbf{a}_1\mathbf{a}_1^T\\
                =&(\mathbf{a}_2\Sigma\mathbf{a}_1)^T-\lambda(\mathbf{a_2}\mathbf{a}_1)^T-\phi\lambda_1=0
            \end{align*}
            So $\phi$ must be zero. Hence
            \[
                \Sigma\mathbf{a}_2^T = \lambda\mathbf{a}_2^T
            \]
            $\mathbf{a}_2^T$ is also an eigenvector and has $\lambda\equiv\lambda_2$ as its eigenvalue. And, again, $var[\mathbf{z}_2]=\mathbf{a}_2\Sigma\mathbf{a}_2^T=\lambda_2$ is the second largest eigenvalue.\\[0.2cm]
            In general,
            \[
                var[\mathbf{z}_k]=\mathbf{a}_k\Sigma\mathbf{a}_k^T=\lambda_k
            \]
            The $k^\mathrm{th}$ largest eigenvalue of $\Sigma$ is the variance of the $k^\mathrm{th}$ PC.
\end{document}
