\documentclass[../main.tex]{subfiles}

\begin{document}
    \section{Basic Statistic Notions}
        \subsection{Expected Value \texorpdfstring{$E[X]=\mu$}{}}
            \[
                E[X]=\left\{\begin{array}{ll}
                        \sum x_ip_i& \mathrm{if~discrete} \\[0.2cm]
                        \int_{\mathbb{R}}xf(x)dx & \mathrm{if~continuous}
                \end{array}\right.
            \]
        \subsection{Variance \texorpdfstring{$var[X]=\sigma_X^2$}{}}
            \[
                var[X]=\left\{\begin{array}{ll}
                        \sum (x_i-\mu)^2p_i& \mathrm{if~discrete} \\[0.2cm]
                        \int_{\mathbb{R}}(x-\mu)^2f(x)dx & \mathrm{if~continuous}
                \end{array}\right.
            \]
            If $X$ is continuous, $var[X]=E[(X-E[X])^2]=E[X^2] - E[X]^2$.
        \subsection{Standard Deviation \texorpdfstring{$\sigma_X$}{}}
            \[
                \sigma_X=\left\{\begin{array}{ll}
                        \sqrt{\sum(x_i-\mu)^2p_i}& \mathrm{if~discrete} \\[0.2cm]
                        \sqrt{\int_{\mathbb{R}}(x-\mu)^2f(x)dx} & \mathrm{if~continuous}
                \end{array}\right.
            \]
            Note that $\sigma_X=\sqrt{var[X]}$. If $X$ is discrete and each $p_i=\frac{1}{N}$, $\sigma_X=\sqrt{\frac{\sum(x_i-\mu)^2}{N}}$.
        \subsection{Covariance \texorpdfstring{$cov[X,Y]$}{}}
            \subsubsection{Two Variables}
                $X,Y$ are random variables with space $\mathbb{R}$.
                \[
                cov[X,Y]=\left\{\begin{array}{ll}
                        \sum p_i(x_i-\mu_X)(y_i-\mu_Y)& \mathrm{if~discrete} \\[0.2cm]
                        \int_{\mathbb{R}\times\mathbb{R}}(x-\mu_X)(y-\mu_Y)f(x,y) & \mathrm{if~continuous}
                \end{array}\right.
                \]
                Note that $cov[X,X]=var[X]$ and covariance matrix
                $
                    \Sigma = \begin{pmatrix}
                        cov[X,X] & cov[X,Y]\\
                        cov[Y,X] & cov[Y,Y]
                    \end{pmatrix}
                $
            \subsubsection{More Variables}
                $X_1,X_2,\dots,X_n$ are random variables with space $\mathbb{R}$.\\
                The covariance matrix $\Sigma$ is
                \[
                    \Sigma = \begin{pmatrix}
                        cov[X_1,X_1] & cov[X_1,X_2] & \cdots & cov[X_1,X_n]\\
                        cov[X_2,X_1] & cov[X_2,X_2] & \cdots & cov[X_2,X_n]\\
                        \vdots & \vdots & \ddots & \vdots\\
                        cov[X_n,X_1] & cov[X_n,X_2] & \cdots & cov[X_n,X_n]
                    \end{pmatrix}
                \]
                Or let $X=(X_1,X_2,\dots,X_n)$, $\Sigma=cov[X,X]=E[(X-E[X])(X-E[X])^T]$.
                
        \subsection{Correlation \texorpdfstring{$corr[X,Y]=\rho_{X,Y}$}{}}
            $corr[X,Y]=\frac{cov[X,Y]}{\sigma_X \sigma_Y}\in[-1,1]$.
        \subsection{Assume sample come from one distribution}
            \subsubsection{Gaussian Distribution}
                \[
                    f_{\mu,\Sigma}(x)=\frac{1}{(2\pi)^{D/2}}\frac{1}{|\Sigma|^{1/2}}e^{-\frac{1}{2}(x-\mu)^T\Sigma^{-1}(x-\mu)}
                \]
                where $x$ vector, $D$ for dimension, $\mu$ mean vector, $\Sigma$ covariance matrix.\\
                An unknown $f_{\mu,\Sigma}(x)$ samples $n$ points $x_1,\dots,x_n$. The likelihood function
                \[
                    L(\mu,\Sigma)=f_{\mu,\Sigma}(x_1)f_{\mu,\Sigma}(x_2)\cdots f_{\mu,\Sigma}(x_n)
                \]
                Higher value of the likelihood function means higher probability to sample $x_1,\dots,x_n$. The maximum likelihood estimator $(\mu^*,\Sigma^*)$ is
                \[
                    \mu^*,\Sigma^* = \argmax_{\mu,\Sigma} L(\mu,\Sigma)
                \]
                Also, for $l(\mu,\Sigma)=\ln L(\mu,\Sigma)$,
                \[
                    \mu^*,\Sigma^* = \argmax_{\mu,\Sigma} l(\mu,\Sigma)
                \]
                Then Gaussian distribution
                \[\mu^* = \frac{1}{n}\sum_1^n x_i,\quad \Sigma^* = \frac{1}{n}\sum_1^n(x_i-\mu^*)(x_i-\mu^*)^T\]
    
        \subsection{Two Classes \texorpdfstring{$C_1,C_2$}{}}
            $C_1$ has $\{x_i\}_1^{70}$, $C_2$ has $\{x_i\}_{71
            }^{180}$. The posterior probability of an unknown object $x$ is
            \[
                P(C_1|x) = \frac{P(x|C_1)P(C_1)}{P(x|C_1)P(C_1)+P(x|C_2)P(C_2)}
            \]
            The terms $P(C_k)$ are easy to get. The terms $P(x|C_k)$ is based on the distribution we assumed. If it's Gaussian,
            \[
                P(x|C_1) = f_{\mu_1^*,\Sigma_1^*}(x),\quad P(x|C_2) = f_{\mu_2^*,\Sigma_2^*}(x)
            \]
            We can modify $\Sigma^*$ by
            \[
                \Sigma^* = \frac{70}{70+110}\Sigma_1^* + \frac{110}{70+110}\Sigma_2^*
            \]
            And let $C_1,C_2$ share this $\Sigma^*$, the likelihood function becomes
            \[
                L(\mu_1,\mu_2,\Sigma) = f_{\mu_1,\Sigma}(x_1)\cdots f_{\mu_1,\Sigma}(x_{70})f_{\mu_2,\Sigma}(x_{71})\cdots f_{\mu_2,\Sigma}(x_{180})
            \]
            $\mu_1^*,\mu_2^*,\Sigma^*$ are maximum likelihood estimators.
            The new model will be a linear classifier. Explain in the following.

            Let $z=\ln \frac{P(x|C_1)P(C_1)}{P(x|C_2)P(C_2)}$. Then
            \[
                P(C_1|x)=\frac{1}{1+e^{-z}}=\sigma(z),\quad\mathrm{the~sigmoid~function}.
            \]
            $P(x|C_k)=f_{\mu_k^*,\Sigma^*}(x)$, $k=1,2$. So
            \[
                z=\ln\frac{f_{\mu_1^*,\Sigma^*}(x)}{f_{\mu_2^*,\Sigma^*}(x)}+\ln\frac{P(C_1)}{P(C_2)}
            \]
            Put the details in and compute, we get
            \[
                z = (\mu_1^*-\mu_2^*)^T\Sigma^{*-1}x-\frac{1}{2}(\mu_1^*)^T\Sigma^{*-1}\mu_1^*+\frac{1}{2}(\mu_2^*)^T\Sigma^{*-1}\mu_2^*+\ln\frac{P(C_1)}{P(C_2)}
            \]
            Let $w^T=(\mu_1^*-\mu_2^*)^T\Sigma^{*-1}$, $b=-\frac{1}{2}(\mu_1^*)^T\Sigma^{*-1}\mu_1^*+\frac{1}{2}(\mu_2^*)^T\Sigma^{*-1}\mu_2^*+\ln\frac{P(C_1)}{P(C_2)}$. Then
            \[
                P(C_k|x)=\sigma(w \cdot x+b)
            \]
\end{document}