\documentclass{article}
\usepackage[a4paper]{geometry}
\usepackage[utf8]{inputenc} 
\usepackage{indentfirst}
%list code
\usepackage{listings}
%content include references
\usepackage[nottoc,numbib]{tocbibind}
% table of contents clickabel
\usepackage{hyperref}
% make file system neat and easier to manage
\usepackage{subfiles}
% packages
\usepackage{csquotes}
\usepackage{arydshln}
\usepackage{dsfont}
\usepackage{polynom}
\usepackage{empheq}
\usepackage{calc}  
\usepackage{enumitem}
\usepackage{graphicx}
\usepackage{systeme}
\usepackage{mathtools}
\usepackage{tikz}
\usepackage{amsfonts, mathrsfs, bm, amsmath, amssymb, bbm, amsthm}
\usepackage{ifthen}
\usepackage{enumerate}
% Make the preview part more eye friendly
\usepackage{xcolor}
\pagecolor[rgb]{1,1,1} % background
\color[rgb]{0,0,0} % foreground
% new command
\DeclareMathOperator*{\argmax}{arg~max}
\DeclareMathOperator*{\argmin}{arg~min}
% theorem block
\newtheorem{theorem}{Theorem}[section]
\newtheorem{note}{Note}
% cross page equation
\allowdisplaybreaks

\begin{document}
    \section{Tuning Learning Rate(Optimizer)}
        \subsection{Momentum}
            Let $v_k$ be the variable that stores previous move, i.e. the momentum. In the beginning, $v_0=0$.
            \begin{align*}
                \mathrm{Initialize~} & \theta_0 \mathrm{~and~let~} v_0=0. \\
                & \theta_1 = \theta_0 + v_1,\quad v_1 = \lambda v_0 - \alpha \nabla L(\theta_0)=- \alpha \nabla L(\theta_0), \\
                & \theta_2 = \theta_1 + v_2,\quad v_2 = \lambda v_1 - \alpha \nabla L(\theta_1)=-\lambda\alpha \nabla L(\theta_0)- \alpha \nabla L(\theta_1), \\
                         &\quad \vdots\\
                & \theta_{t+1} = \theta_{t} + v_{t+1},\quad v_{t+1} = \lambda v_{t} - \alpha \nabla L(\theta_{t})
            \end{align*}
            Briefly, momentum method perturb current gradient by previous gradient(momentum).

        \subsection{Nesterov Accelerated Gradient(NAG)}
            Similar to momentum
            \begin{align*}
                \mathrm{Initialize~} & \theta_0 \mathrm{~and~let~} v_0 = 0. \\
                & \theta_1 = \theta_0 + v_1, \quad v_1 = \lambda v_0 - \alpha \nabla L(\theta_0 + \lambda v_0) = -\alpha \nabla L(\theta_0), \\
                & \theta_2 = \theta_1 + v_2, \quad v_2 = \lambda v_1 - \alpha \nabla L(\theta_1 + \lambda v_1) = -\lambda \alpha \nabla L(\theta_0) - \alpha \nabla L(\theta_1 + \lambda v_1), \\
                         &\quad \vdots\\
                & \theta_{t+1} = \theta_t + v_{t+1}, \quad v_{t+1} = \lambda v_t - \alpha \nabla L(\theta_t + \lambda v_t)
            \end{align*}
            Here, instead of perturbing current gradient, we perturb current parameter by previous gradient.
            
        \subsection{Adagrad(Adaptive Gradient)}
            Use first derivative to estimate second derivative.
            \begin{align*}
                & \alpha\leftarrow\frac{\alpha}{\sqrt{\sum_{i=0}^t\big(\nabla L(\theta_i)\big)^2}}, \\
                & \theta_t \leftarrow \theta_{t-1} - \frac{\alpha}{\sqrt{\sum_{i=0}^t\big(\nabla L(\theta_i)\big)^2}}\nabla L(\theta^{t-1}) 
            \end{align*}
            In practice we will add $\epsilon$ in the denominator to avoid dividing by zero
            \[
                \theta_t \leftarrow \theta_{t-1} - \frac{\alpha}{\sqrt{ \sum_{i=0}^t \big( \nabla L(\theta_i) \big)^2 + \epsilon}}\nabla L(\theta^{t-1})        
            \]
        \subsection{Adadelta}
            This method needs the average of gradients $\mathrm{E}[\nabla L(\theta)^2]_t$ at step $t$ and the average of parameter update $\mathrm{E}[\Delta \theta^2]_t$. Initialize $\mathrm{E}[\nabla L(\theta)^2]_0 = 0,~\mathrm{E}[\Delta \theta^2]_0 = 0$ and choose a decay rate $\rho$, learning rate $\alpha$ and small $\epsilon$.
            \begin{align*}
                \mathrm{I}.~& \theta_1 = \theta_0 + \Delta \theta_0,~ \Delta \theta_0 = - \frac{\alpha}{ \sqrt{ \mathrm{E}[\nabla L(\theta)^2]_0 + \epsilon} }\nabla L(\theta_0) \\[0.3cm]
                \mathrm{II}.~& \mathrm{E}[\Delta \theta^2]_0 = 0\\
                \mathrm{II}.~& \mathrm{E}[\nabla L(\theta)^2]_1 = \rho \mathrm{E}[\nabla L(\theta)^2]_0 + (1 - \rho) \nabla L(\theta_1)^2\\
                \mathrm{II}.~& \theta_2 = \theta_1 + \Delta \theta_1,~ \Delta \theta_1 = - \frac{ \sqrt{ \mathrm{E}[\Delta \theta^2]_0 + \epsilon} }{ \sqrt{\mathrm{E}[\nabla L(\theta)^2]_1 + \epsilon} }\nabla L(\theta_1) \equiv - \frac{RMS[\Delta \theta]_0}{RMS[\nabla L(\theta)]_1}\nabla L(\theta_1) \\[0.3cm]
                \mathrm{III}.~& \mathrm{E}[\Delta \theta^2]_1 = \rho \mathrm{E}[\Delta \theta^2]_0 + (1-\rho) \Delta\theta_1^2 \\
                \mathrm{III}.~& \mathrm{E}[\nabla L(\theta)^2]_2 = \rho \mathrm{E}[\nabla L(\theta)^2]_1 + (1 - \rho) \nabla L(\theta_2)^2 \\
                \mathrm{III}.~& \theta_3 = \theta_2 + \Delta \theta_2,~ \Delta \theta_2 = - \frac{RMS[\Delta \theta]_1}{RMS[\nabla L(\theta)]_2}\nabla L(\theta_2)\\[0.3cm]
                \mathrm{\#}.~& \mathrm{E}[\Delta \theta^2]_{t-2} = \rho \mathrm{E}[\Delta \theta^2]_{t-3} + (1-\rho) \Delta\theta_{t-2}^2\\
                \mathrm{\#}.~& \mathrm{E}[\nabla L(\theta)^2]_t = \rho \mathrm{E}[\nabla L(\theta)^2]_{t-1} + (1 - \rho) \nabla L(\theta_t)^2\\
                \mathrm{\#}.~& \theta_t = \theta_{t-1} + \Delta \theta_{t-1},~\Delta \theta_{t-1} = - \frac{RMS[\Delta \theta]_{t-2}}{RMS[\nabla L(\theta)]_{t-1}}\nabla L(\theta_{t-1})
            \end{align*}
        \subsection{RMSprop(Root Mean Square Propagation)}
            Manually determine a weight $\beta$.
            \begin{align*}
                \theta_1 & \leftarrow \theta_0 - \frac{\alpha}{\sigma_0}\nabla L(\theta_{0}),\quad\sigma_0 = \nabla L(\theta_{0}), \\
                \theta_2 & \leftarrow \theta_1 - \frac{\alpha}{\sigma_2}\nabla L(\theta_{1}),\quad\sigma_1 = \sqrt{\beta(\sigma_0)^2 + (1-\beta)\big(\nabla L(\theta_{1})\big)^2 + \epsilon}, \\
                         &\quad\vdots\\
                \theta_{t+1} & \leftarrow \theta_t - \frac{\alpha}{\sigma_t}\nabla L(\theta_{t}),\quad\sigma_t = \sqrt{\beta(\sigma_{t-1})^2+(1-\beta)\big(\nabla L(\theta_{t})\big)^2 + \epsilon}
            \end{align*}
        \subsection{Adam(RMSprop+Momentum)}
            Two weight numbers $\beta_1$ and $\beta_2$.  Two moment vectors $v_k$ and $\sigma_k$. In the beginning $v_0=0$ and $\sigma_0=0$.
            \begin{align*}
                &\theta_1 = \theta_0 - \alpha\frac{\sigma_1}{\sqrt{v_1}+\epsilon},\quad\sigma_1 = \frac{\beta_1\sigma_0+(1-\beta_1)\nabla L(\theta_0)}{1-\beta_1},\quad v_1=\frac{\beta_2 v_0+(1-\beta_2)\big(\nabla L(\theta_0)\big)^2}{1-\beta_2}, \\
                &\theta_2 = \theta_1 - \alpha\frac{\sigma_2}{\sqrt{v_2}+\epsilon},\quad\sigma_2=\frac{\beta_1\sigma_1+(1-\beta_1)\nabla L(\theta_1)}{1-\beta_1^2},\quad v_2 = \frac{\beta_2 v_1+(1-\beta_2)\big(\nabla L(\theta_1)\big)^2}{1-\beta_2^2}, \\
                    &\qquad\vdots\\
                &\theta_{t+1}=\theta_t-\alpha\frac{\sigma_{t+1}}{\sqrt{v_{t+1}}+\epsilon},\quad\sigma_{t+1}=\frac{\beta_1\sigma_t+(1-\beta_1)\nabla L(\theta_t)}{1-\beta_1^t},\quad v_{t+1}=\frac{\beta_2 v_t+(1-\beta_2)\big(\nabla L(\theta_t)\big)^2}{1-\beta_2^t}
            \end{align*}
        
\end{document}
