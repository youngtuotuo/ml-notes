\documentclass{article}
\usepackage[a4paper]{geometry}
\usepackage[utf8]{inputenc} 
\usepackage{indentfirst}
%list code
\usepackage{listings}
%content include references
\usepackage[nottoc,numbib]{tocbibind}
% table of contents clickabel
\usepackage{hyperref}
% make file system neat and easier to manage
\usepackage{subfiles}
% packages
\usepackage{csquotes}
\usepackage{arydshln}
\usepackage{dsfont}
\usepackage{polynom}
\usepackage{empheq}
\usepackage{calc}  
\usepackage{enumitem}
\usepackage{graphicx}
\usepackage{systeme}
\usepackage{mathtools}
\usepackage{tikz}
\usepackage{amsfonts, mathrsfs, bm, amsmath, amssymb, bbm, amsthm}
\usepackage{ifthen}
\usepackage{enumerate}
% Make the preview part more eye friendly
\usepackage{xcolor}
\pagecolor[rgb]{1,1,1} % background
\color[rgb]{0,0,0} % foreground
% new command
\DeclareMathOperator*{\argmax}{arg~max}
\DeclareMathOperator*{\argmin}{arg~min}
% theorem block
\newtheorem{theorem}{Theorem}[section]
\newtheorem{note}{Note}
% cross page equation
\allowdisplaybreaks

\begin{document}
    \section{Information Theory}
        \subsection{Information}

            \begin{displayquote}
                \textit{Highly improbable events bring more information to us,
                while certain events bring no information.}
            \end{displayquote}
            
            The information of an event $ x $ will therefore depends on the
            probability distribution $ p(x) $ of its random variable $ X $.

        \subsection{Construct Information Fomula}

            Let $ h(\cdot)$ be a monotonic function of any distribution $ p(x)
            $ that returns the information of $ p(x) $. If $ x $ and $ y $ are
            unrelated events, we hope the information they take are also
            unrelated, so

            \begin{align}
                &h(x,y) = h(x) + h(y) \label{eq:1} \\
                &p(x,y) = p(x) p(y) \label{eq:2}
            \end{align}

            Note that we can interpret $ h( p(x) ) $ as $ h( x ) $ and
            interpret $ h( p(x, y) ) $ as $ h(x, y) $. 

            $ \log_{2}( x ) $ is a monotonic function that satisfied both
            (\ref{eq:1}) and (\ref{eq:2}), hence we can define

            \begin{equation*}
                h(x) = - \log_2 p(x)
            \end{equation*}

            Then $ h(x) $ satisfies $ 2^{ h(x) } = \frac{1}{p(x)} $. We can
            interpret this as

            \begin{displayquote}
                $ h(x) $ \textit{is the amount of bits that being enough for
                representing} $ \frac{1}{p(x)} $ \textit{in binary}.
            \end{displayquote}

            When $ p(x) $ is low, the probability is low, we need more bits to
            represent it.

        \subsection{Entropy}

            Let $ X $ be the random variable of the state that transmitted from
            a sender to a receiver. Intuitively, \textit{the average amount of
            the information} that $ X $ carries is obtained by taking the
            expectation of information $ h(x) $ with respect to the p.d.f.
            $ p(x) $

            \begin{equation*}
                 \sum_{x \in X} p(x) h(x) = - \sum_{x \in X} p(x) \log_2 p(x)
            \end{equation*}

            This is called the \textit{entropy} of the random variable $ X $
            and denote it by $ \operatorname{H}(X) $ or $ \operatorname{H}(p) $
            or $ \operatorname{H}(x) $, based on the context of the paragraph.

            Since $ \lim_{p \rightarrow 0} p \ln p = 0 $,
            we just take $ p(x) \ln p(x) = 0 $ when we encounter $ p(x) = 0 $ for some
            $ x $.
            
            \subsubsection{Nats}

                In practice, we use $ \ln p(x)$ instead of $\log_2 p(x) $. That is,

                \begin{align*}
                    &h(x)=-\ln p(x)\\
                    &\operatorname{H}(p) = -\sum_{x \in X} p(x) \ln p(x)
                \end{align*}

                In this situation, we said the information is measured in the
                units of 'nats'.

            \subsubsection{Entropy as Lower Bound}

                \begin{displayquote}
                    \textit{Entropy is a lower bound of the amount of bits that
                    a random variable can transmits.}
                \end{displayquote}

                by the \textit{Noiseless Coding Theorem}.

            \subsubsection{Noiseless Coding Theorem}

                \begin{displayquote}
                     $ N $ i.i.d. random variables each with entropy $ \operatorname{H}(X) $
                     can be compressed into more than $ N \operatorname{H}(X) $ bits with
                     negligible risk of information loss, as $ N \rightarrow
                     \infty $; but conversely, if they are compressed into
                     fewer than $ N \operatorname{H}(X) $ bits it is virtually certain that
                     information will be lost.
                \end{displayquote}

                (ChatGPT) In essence, the theorem states that for any given
                data source with a certain probability distribution of symbols,
                it is possible to encode the source in such a way that the
                average length of the encoded message per symbol is close to
                the entropy of the source.

        \subsection{Maximize Entropy in Discrete Case}

            \subsubsection*{TL;DR}

                \begin{displayquote}
                    \textit{The distribution that can carry the most average
                    amount of information is the uniform.}
                \end{displayquote}

            \subsubsection*{Proof}
            
                Let $ X = \{x_i\}_{ i=1 }^M $ be a discrete random variable and let $ p $
                be the distribution of $ X $. For the optimization problem

                \begin{equation*}
                     \max \left( - \sum_{i=1}^M p(x_i) \ln p(x_i) \right)
                \end{equation*}

                with the normalization constraint on the probabilities

                \begin{equation*}
                     \sum_{i=1}^M p(x_i) = 1
                \end{equation*}

                The Lagrangian is

                \begin{equation*}
                     \mathcal{L} = -\sum_{i=1}^M p(x_i)\ln p(x_i) +
                     \lambda\left(\sum_{i=1}^M p(x_i) - 1\right)
                \end{equation*}

                From $ \partial \mathcal{L} / \partial p(x_i) = - ( \ln p(x_i) + 1) +
                \lambda = 0 $, we have $ \lambda = \ln p(x_i) + 1 $,
                $ \forall i = 1, \dots, M $. By $ \sum_{i=1}^M p(x_{i}) = 1 $ and $
                \lambda = \ln p(x_i) + 1 $, it's easy to get $ \lambda = \ln(1 / M)
                + 1 $, we then have 

                \begin{equation*}
                     p(x_i) = \frac{1}{M},\forall i
                \end{equation*}

                is the stationary point.

                To verify the maximum, first compute the Hessian matrix

                \begin{equation*}
                     \frac{\partial \mathcal{L}}{\partial p(x_{i}) \partial
                     p(x_{j})} = - I_{ij} \frac{1}{p(x_i)}
                \end{equation*}

                It's obvious that all the eigenvalues are negative (negative
                definite). So $ p(x_{i}) = \frac{1}{M} $ actually attains a
                maximum, and the maximum entropy is $ \operatorname{H}(p) = \ln M $.

            

            
        \subsection{Differential Entropy}

            Let $ X \subseteq \mathbb{R} $ be a continuous random variable and $ p(x) $ be the
            distribution of $ X $. By M.V.T, we know there exists some $ x_i $
            such that

            \begin{equation*}
                 \int_{i \Delta}^{ (i+1)\Delta }p(x) dx = p(x_i) \Delta
            \end{equation*}

            where $ \Delta $ is the length of one partition of $ X $. Now for any
            $ x \in [i \Delta, (i + 1) \Delta]$, we can use $ p(x_i) \Delta $ to estimate
            its probability as long as $ \Delta $ small enough. Here comes an
            entropy estimation

            \begin{align*}
                \operatorname{H}_{\Delta} &= - \sum_i p(x_i) \Delta \ln( p(x_i) \Delta) \\
                      &= - \sum_i p(x_i) \Delta \ln(p(x_i) \Delta) + \sum_i p(x_i) \Delta \ln \Delta - \sum_i p(x_i) \Delta \ln \Delta \\
                      &= - \sum_i p(x_i) \Delta \ln \left( \frac{p(x_i) \Delta}{\Delta} \right) - \left( \sum_i p(x_i) \Delta \right) \ln \Delta \\
                      &= - \sum_i p(x_i) \Delta \ln p(x_i) - \ln \Delta
            \end{align*}

            Note that $ \sum_i p(x_i) \Delta = \int_{x \in X} p(x) dx = 1 $.

            The limit of the first term of right hand side is

            \begin{equation*}
                 \lim_{\Delta \rightarrow 0} - \sum_i p(x_i) \Delta \ln p(x_i)
                 = - \int p(x) \ln p(x) dx
            \end{equation*}

            This integral is called the \textit{differential entropy}.

            The difference term $ \ln \Delta $ shows the fact that we need lots of bits to
            describe a continuous variable.

            The differential entropy can have negative values when $ \sigma^2 < (1 / 2 \pi e) $.

            For multi dimension random variable, the differential entropy is similar

            \begin{equation*}
                \operatorname{H}(p) = - \int p(\mathbf{x}) \ln p(\mathbf{x}) d \mathbf{x} 
            \end{equation*}

        \subsection{Conditional Entropy}

            Given a joint probability $ p(\mathbf{x}, \mathbf{y}) $. When $
            \mathbf{x} $ is known, the additional information needed to specify
            the corresponding value of $ \mathbf{y} $ is given by $ - \ln
            p(\mathbf{y} | \mathbf{x}) $. We can compute the \textit{conditional entropy}

            \begin{equation*}
                \operatorname{H}(\mathbf{y} | \mathbf{x}) = 
                 - \iint p(\mathbf{x}, \mathbf{y}) \ln p( \mathbf{y} | \mathbf{x} ) d \mathbf{y} d \mathbf{x}
            \end{equation*}

            Note that

            \begin{align*}
                \operatorname{H}(\mathbf{x}, \mathbf{y})
                      &= - \iint p(\mathbf{x}, \mathbf{y})
                         \ln p(\mathbf{x}, \mathbf{y}) d \mathbf{x} d \mathbf{y} \\
                      &= - \iint p(\mathbf{x}, \mathbf{y}) 
                         \ln ( p(\mathbf{y} | \mathbf{x}) p(\mathbf{x})) d \mathbf{x} d \mathbf{y} \\
                      &= - \iint p(\mathbf{x}, \mathbf{y}) \ln p(\mathbf{y} | \mathbf{x}) d \mathbf{x}
                          d \mathbf{y} - \iint p(\mathbf{x}, \mathbf{y}) \ln p(\mathbf{x}) d \mathbf{x}
                          d \mathbf{y} \\
                      &= - \iint p(\mathbf{x}, \mathbf{y}) \ln p(\mathbf{y} | \mathbf{x}) d \mathbf{x}
                          d \mathbf{y} - \int \left( \int p(\mathbf{x}, \mathbf{y}) d \mathbf{y} \right)
                          \ln p(\mathbf{x}) d \mathbf{x} \\
                      &= \operatorname{H} (\mathbf{y} | \mathbf{x}) + \operatorname{H} (\mathbf{x})
            \end{align*}

        \subsection{Cross Entropy}

            The cross entropy of the distribution $ q(x) $ relative to a
            distribution $ p(x) $ is

            \begin{equation*}
                H(p,q) = -\mathrm{E}_p[\ln q] = - \sum_x p(x) \ln q(x) 
            \end{equation*}

            In deep learning, $ p(x) $ often refers to the ground truth label,
            and $ q(x) $ refers to the output from a deep neural network model.

            In information theory, minimize cross entropy means

            \begin{displayquote}
                 \textit{Minimizes the amount of information required to
                 specify the value of} $ x $ \textit{as a result of using} $
                 q(x) $.
            \end{displayquote}
        
        \subsection{Kullback-Leibler Divergence}

            Let $ p(x) $ be an unknown distribution and we use $ q(x) $ to
            approximate it. This will cause additional amount of information

            \begin{align*}
                \operatorname{KL}(p \| q) &= \left(
                        -\int p(\mathbf{x}) \ln
                        q(\mathbf{x}) d \mathbf{x}
                    \right) - \left(
                        -\int p(\mathbf{x}) \ln
                        p(\mathbf{x}) d \mathbf{x}
                    \right) \\
                                 &= -\int p(\mathbf{x}) \ln
                    \left\{
                        \frac{ q(\mathbf{x}) }{ p(\mathbf{x}) }
                    \right\} d \mathbf{x}
            \end{align*}

            This is known as the \textit{KL divergence} from $ q $ to $ p $.
            \textit{KL divergence} is also called the \textit{relative entropy}.

            Note that $ \operatorname{KL}(p \| q) \neq \operatorname{KL}(q \| p) $.

            \subsubsection*{TL;DR}

                $ \operatorname{KL}(p \| q) \geq 0 $, $ \operatorname{KL}(p \| q) = 0 \iff p = q $.


            \subsubsection*{Convex function}
                
                For $ 0 \leq \lambda \leq 1 $, a convex function satisfies

                \begin{equation*}
                    f(\lambda a + (1 - \lambda) b) \leq \lambda f(a) + (1 - \lambda) f(b)
                \end{equation*}

                A convex function is called strictly convex if the equality
                holds only when $ \lambda = 0 $ or $ \lambda = 1 $.

            \subsubsection*{Jensen's Inequality}

                Recall the \textit{Jensen's inequality}, for a convex function $ f $,

                \begin{equation*}
                    f \left( \sum_{i} \lambda_{i} x_{i}  \right) \leq \sum_{i} \lambda_{i} f(x_{i})
                \end{equation*}

                where $ \lambda_{i} \geq 0 $ and $ \sum_{i} \lambda_{i} = 1 $.

                When each $ \lambda_{i} $ becomes the probability $ p(x_{i}) $, we have

                \begin{equation*}
                    f\left( \operatorname{E}[x] \right) \leq \operatorname{E}[f(x)].
                \end{equation*}

                For continuous random variable, we have

                \begin{equation*}
                     f\left( \int \mathbf{x} p(\mathbf{x}) d \mathbf{x} \right)
                     \leq \int f(\mathbf{x}) p(\mathbf{x}) d \mathbf{x}
                \end{equation*}

            \subsubsection*{Proof}

                Observe that $ - \ln x $ is a strictly convex function, so by Jensen's inequality

                \begin{align*}
                    \operatorname{KL}(p \| q)
                        &= - \int p(\mathbf{x}) \ln
                     \frac{q(\mathbf{x})}{p(\mathbf{x})} d \mathbf{x} \\
                        &\geq - \ln \left( \int p(\mathbf{x}) \frac{q(\mathbf{x})}{p(\mathbf{x})} d \mathbf{x} \right) \\
                        &=    - \ln \left( \int q(\mathbf{x}) d \mathbf{x} \right) = 0
                \end{align*}

                When the equality holds,

                \begin{align*}
                                & p(\mathbf{x}) \ln \frac{q(\mathbf{x})}{p(\mathbf{x})} = 0 \\
                    \Rightarrow & \ln \frac{q(\mathbf{x})}{p(\mathbf{x})} = 0 \\
                    \Rightarrow & \frac{q(\mathbf{x})}{p(\mathbf{x})} = 1 \\
                    \Rightarrow & q(\mathbf{x}) = p(\mathbf{x})
                \end{align*}

        \subsection{Gaussian Maximizes Differential Entropy}

            We want to maximize
            
            \begin{equation*}
                 \operatorname{H}(p) = - \int p(x) \ln p(x) dx
            \end{equation*}

            with three natural constraints

            \begin{align*}
                &\int_{-\infty}^\infty p(x) dx = 1 \\
                &\int_{-\infty}^\infty x p(x) dx = \mu \\
                &\int_{-\infty}^\infty (x-\mu)^2 p(x) dx = \sigma^2
            \end{align*}

            The Lagrangian is

            \begin{multline*}
                 \mathcal{L}[p] = -\int_{\mathbb{R}} p(x) \ln p(x) dx +
                 \lambda_1 \left( \int_{\mathbb{R}} p(x) dx - 1 \right) + \\
                 \lambda_2 \left( \int_{\mathbb{R}} x p(x) dx - \mu \right) +
                 \lambda_3 \left( \int_{\mathbb{R}} (x - \mu)^2 p(x) dx - \sigma^2 \right)
            \end{multline*}

            This is a functional. The derivative of a functional is
            denoted by $ \frac{\delta \mathcal{L}}{\delta p} $ and is defined to satisfy

            \begin{equation*}
                 \int \frac{\delta \mathcal{L}[p]}{\delta p} \phi(x) dx =
                 \lim_{\epsilon \rightarrow 0}\frac{\mathcal{L}[p + \epsilon
                 \phi] - \mathcal{L}[p]}{\epsilon} = 
                 \left[
                     \frac{d}{d \epsilon} \mathcal{L}[p + \epsilon \phi]
                 \right]_{\epsilon = 0}
            \end{equation*}

            where $ \phi(x) $ is a variation term.

            Deriving from the definition

            \begin{equation*}
                 \int \frac{\delta \mathcal{L}[p]}{\delta p} \phi(x) dx = \int 
                 \left( - ( \ln p(x) + 1 ) + \lambda_1 + \lambda_2 x +
                 \lambda_3 (x - \mu)^2 \right) \phi(x) dx
            \end{equation*}

            We have the actual form of $ \frac{\delta \mathcal{L}[p]}{\delta p} $ and can let it be zero then get

            \begin{equation*}
                 p(x) = e^{ -1 + \lambda_1 + \lambda_2 x + \lambda(x - \mu)^2 }
            \end{equation*}

            Substitute this result back to three constraints leading to

            \begin{equation*}
                 p(x) = \frac{1}{ (2 \pi \sigma^2)^{\frac{1}{2}} } e^{ -\frac{(x - \mu)^2}{2 \sigma^2} }
            \end{equation*}

            which is the Gaussian.

            To verify the maximum, let $ f(x) $ be any distribution has the
            variance $ \sigma^2 $. Since differential entropy is translation invariant, we can also assume
            $ f(x) $ has the same mean $ \mu $. Now consider the KL divergence

            \begin{align*}
                & \operatorname{KL}(f \| p) = -\int f(x) \ln \left( \frac{p(x)}{f(x)} \right) dx \\
                =& -\operatorname{H}(f) - \int f(x) \ln \left(
                     \frac{1}{ (2 \pi
                     \sigma^2)^{\frac{1}{2}} } e^{
                 -\frac{(x - \mu)^2}{2 \sigma^2} } 
                \right) dx \\
                =& -\operatorname{H}(f) + \frac{1}{2} \ln (2 \pi \sigma^{2}) +
                \frac{1}{2 \sigma^{2}} \int f(x) (x - \mu)^{2} dx \\
                =& -\operatorname{H}(f) + \frac{1}{2} \ln (2 \pi \sigma^{2}) + \frac{\sigma^{2}}{2 \sigma^{2}} \\
                =& -\operatorname{H}(f) + \frac{1}{2} \ln (2 \pi \sigma^{2}) + \frac{1}{2} \\
                =& -\operatorname{H}(f) + \operatorname{H}(p) \geq 0
            \end{align*}

            Hence

            \begin{equation*}
                 H(p) \geq H(f), \forall f
            \end{equation*}

            The corresponding maximum entropy is,

            \begin{equation*}
                 \operatorname{H}(p) = \frac{1}{2} (1 + \ln(2 \pi \sigma^2))
            \end{equation*}
\end{document}
