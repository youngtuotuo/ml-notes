\documentclass{article}
\usepackage[a4paper]{geometry}
\usepackage[utf8]{inputenc} 
\usepackage{indentfirst}
%list code
\usepackage{listings}
%content include references
\usepackage[nottoc,numbib]{tocbibind}
% table of contents clickabel
\usepackage{hyperref}
% make file system neat and easier to manage
\usepackage{subfiles}
% packages
\usepackage{csquotes}
\usepackage{arydshln}
\usepackage{dsfont}
\usepackage{polynom}
\usepackage{empheq}
\usepackage{calc}  
\usepackage{enumitem}
\usepackage{graphicx}
\usepackage{systeme}
\usepackage{mathtools}
\usepackage{tikz}
\usepackage{amsfonts, mathrsfs, bm, amsmath, amssymb, bbm, amsthm}
\usepackage{ifthen}
\usepackage{enumerate}
% Make the preview part more eye friendly
\usepackage{xcolor}
\pagecolor[rgb]{1,1,1} % background
\color[rgb]{0,0,0} % foreground
% new command
\DeclareMathOperator*{\argmax}{arg~max}
\DeclareMathOperator*{\argmin}{arg~min}
% theorem block
\newtheorem{theorem}{Theorem}[section]
\newtheorem{note}{Note}
% cross page equation
\allowdisplaybreaks

\begin{document}
    \section{K-means clustering}
        Given $N$ data points $\{x_n\}_{n=1}^N\subset\mathbb{R}^D$. Initialize $K$ prototype vectors $\{\mu_k\}_{k=1}^K$. Each $\mu_k$ corresponds to the mean of the $k^\mathrm{th}$ cluster. Let $r_{nk}$ be indicator variable with respect to $x_n$ and $\mu_k$.
        \[
            r_{nk}=\left\{
                \begin{array}{ll}
                    1&\mathrm{if~}k=\argmin\|x_n-\mu_k\|^2\\
                    0&\mathrm{otherwise}
                \end{array}
            \right.
        \]
        Then update $\mu_k$,
        \[
            \mu_k = \frac{\sum_n r_{nk}x_n}{\sum_n r_{nk}}
        \]
        Keep this procedure until
        \[
            J = \sum_{n=1}^N\sum_{k=1}^Kr_{nk}\|x_n-\mu_k\|^2
        \]
        converge.
\end{document}
